\documentclass[
  man,
  floatsintext,
  longtable,
  nolmodern,
  notxfonts,
  notimes,
  colorlinks=true,linkcolor=blue,citecolor=blue,urlcolor=blue]{apa7}

\usepackage{amsmath}
\usepackage{amssymb}



\usepackage[bidi=default]{babel}
\babelprovide[main,import]{english}


% get rid of language-specific shorthands (see #6817):
\let\LanguageShortHands\languageshorthands
\def\languageshorthands#1{}

\RequirePackage{longtable}
\RequirePackage{threeparttablex}

\makeatletter
\renewcommand{\paragraph}{\@startsection{paragraph}{4}{\parindent}%
	{0\baselineskip \@plus 0.2ex \@minus 0.2ex}%
	{-.5em}%
	{\normalfont\normalsize\bfseries\typesectitle}}

\renewcommand{\subparagraph}[1]{\@startsection{subparagraph}{5}{0.5em}%
	{0\baselineskip \@plus 0.2ex \@minus 0.2ex}%
	{-\z@\relax}%
	{\normalfont\normalsize\bfseries\itshape\hspace{\parindent}{#1}\textit{\addperi}}{\relax}}
\makeatother




\usepackage{longtable, booktabs, multirow, multicol, colortbl, hhline, caption, array, float, xpatch}
\setcounter{topnumber}{2}
\setcounter{bottomnumber}{2}
\setcounter{totalnumber}{4}
\renewcommand{\topfraction}{0.85}
\renewcommand{\bottomfraction}{0.85}
\renewcommand{\textfraction}{0.15}
\renewcommand{\floatpagefraction}{0.7}

\usepackage{tcolorbox}
\tcbuselibrary{listings,theorems, breakable, skins}
\usepackage{fontawesome5}

\definecolor{quarto-callout-color}{HTML}{909090}
\definecolor{quarto-callout-note-color}{HTML}{0758E5}
\definecolor{quarto-callout-important-color}{HTML}{CC1914}
\definecolor{quarto-callout-warning-color}{HTML}{EB9113}
\definecolor{quarto-callout-tip-color}{HTML}{00A047}
\definecolor{quarto-callout-caution-color}{HTML}{FC5300}
\definecolor{quarto-callout-color-frame}{HTML}{ACACAC}
\definecolor{quarto-callout-note-color-frame}{HTML}{4582EC}
\definecolor{quarto-callout-important-color-frame}{HTML}{D9534F}
\definecolor{quarto-callout-warning-color-frame}{HTML}{F0AD4E}
\definecolor{quarto-callout-tip-color-frame}{HTML}{02B875}
\definecolor{quarto-callout-caution-color-frame}{HTML}{FD7E14}

%\newlength\Oldarrayrulewidth
%\newlength\Oldtabcolsep


\usepackage{hyperref}




\providecommand{\tightlist}{%
  \setlength{\itemsep}{0pt}\setlength{\parskip}{0pt}}
\usepackage{longtable,booktabs,array}
\usepackage{calc} % for calculating minipage widths
% Correct order of tables after \paragraph or \subparagraph
\usepackage{etoolbox}
\makeatletter
\patchcmd\longtable{\par}{\if@noskipsec\mbox{}\fi\par}{}{}
\makeatother
% Allow footnotes in longtable head/foot
\IfFileExists{footnotehyper.sty}{\usepackage{footnotehyper}}{\usepackage{footnote}}
\makesavenoteenv{longtable}

\usepackage{graphicx}
\makeatletter
\newsavebox\pandoc@box
\newcommand*\pandocbounded[1]{% scales image to fit in text height/width
  \sbox\pandoc@box{#1}%
  \Gscale@div\@tempa{\textheight}{\dimexpr\ht\pandoc@box+\dp\pandoc@box\relax}%
  \Gscale@div\@tempb{\linewidth}{\wd\pandoc@box}%
  \ifdim\@tempb\p@<\@tempa\p@\let\@tempa\@tempb\fi% select the smaller of both
  \ifdim\@tempa\p@<\p@\scalebox{\@tempa}{\usebox\pandoc@box}%
  \else\usebox{\pandoc@box}%
  \fi%
}
% Set default figure placement to htbp
\def\fps@figure{htbp}
\makeatother







\usepackage{newtx}

\defaultfontfeatures{Scale=MatchLowercase}
\defaultfontfeatures[\rmfamily]{Ligatures=TeX,Scale=1}





\title{The relationship between social network size and frontal alpha
asymmetry to strangers during infancy: The moderation of temperament}


\shorttitle{Social Network Size, Frontal Alpha Asymmetry, \&
Temperament}


\usepackage{etoolbox}






\author{Kristen Flaws}



\affiliation{
{MA Program in the Social Sciences, University of Chicago}}




\leftheader{Flaws}



\abstract{Multiple previous studies have reported the close relationship
between children's social network size and social cognition. However,
the connection between social network size and social emotional
development has been less examined. The current study aims to examine
the connection between infants' social network sizes and their neural
response to strangers. We will examine the relationship between social
network sizes and neural responses to strangers during infancy using an
infant's frontal alpha asymmetry (FAA) as a neural indicator.
Additionally, we will investigate how temperament, especially fear, can
moderate the relationship between social network size and responses to
strangers during infancy. Our study will use an Electroencephalogram
(EEG) machine to measure the infant's neural processing of strangers,
specifically alpha wave frequency, and in turn calculate the infant's
FAA while watching videos of strangers. In order to measure an infant's
temperament and social network size, parents will be instructed to fill
out the Revised Infant Behavior Questionnaire (IBQ-R) and Child Social
Network Questionnaire (CSNQ). If we find that a larger social network
size is associated with more positive FAA while viewing the videos of
strangers, we will be further interested in the moderating effect of
fear in this relationship. If we find that fear acts as a moderator, we
will be able to determine if fear increases or decreases the
effectiveness of social network size on infant's responses to strangers.
The findings of this study will contribute to the understanding of the
neurological reactions, and affecting factors, toward strangers during
infancy. }

\keywords{social network size, frontal alpha
asymmetry, fear, temperament, infancy}

\authornote{\par{\addORCIDlink{Kristen Flaws}{0009-0008-7373-4992}} 

\par{       }
\par{Correspondence concerning this article should be addressed
to Kristen Flaws, MA Program in the Social Sciences, University of
Chicago, 1155 E 60th
St., Chicago, IL 60637, USA, Email: kflaws@uchicago.edu}
}

\makeatletter
\let\endoldlt\endlongtable
\def\endlongtable{
\hline
\endoldlt
}
\makeatother

\urlstyle{same}



\usepackage{fontspec}
\usepackage{multirow}
\usepackage{multicol}
\usepackage{colortbl}
\usepackage{hhline}
\newlength\Oldarrayrulewidth
\newlength\Oldtabcolsep
\usepackage{longtable}
\usepackage{array}
\usepackage{hyperref}
\usepackage{float}
\usepackage{wrapfig}
\makeatletter
\@ifpackageloaded{caption}{}{\usepackage{caption}}
\AtBeginDocument{%
\ifdefined\contentsname
  \renewcommand*\contentsname{Table of contents}
\else
  \newcommand\contentsname{Table of contents}
\fi
\ifdefined\listfigurename
  \renewcommand*\listfigurename{List of Figures}
\else
  \newcommand\listfigurename{List of Figures}
\fi
\ifdefined\listtablename
  \renewcommand*\listtablename{List of Tables}
\else
  \newcommand\listtablename{List of Tables}
\fi
\ifdefined\figurename
  \renewcommand*\figurename{Figure}
\else
  \newcommand\figurename{Figure}
\fi
\ifdefined\tablename
  \renewcommand*\tablename{Table}
\else
  \newcommand\tablename{Table}
\fi
}
\@ifpackageloaded{float}{}{\usepackage{float}}
\floatstyle{ruled}
\@ifundefined{c@chapter}{\newfloat{codelisting}{h}{lop}}{\newfloat{codelisting}{h}{lop}[chapter]}
\floatname{codelisting}{Listing}
\newcommand*\listoflistings{\listof{codelisting}{List of Listings}}
\makeatother
\makeatletter
\makeatother
\makeatletter
\@ifpackageloaded{caption}{}{\usepackage{caption}}
\@ifpackageloaded{subcaption}{}{\usepackage{subcaption}}
\makeatother

% From https://tex.stackexchange.com/a/645996/211326
%%% apa7 doesn't want to add appendix section titles in the toc
%%% let's make it do it
\makeatletter
\xpatchcmd{\appendix}
  {\par}
  {\addcontentsline{toc}{section}{\@currentlabelname}\par}
  {}{}
\makeatother

%% Disable longtable counter
%% https://tex.stackexchange.com/a/248395/211326

\usepackage{etoolbox}

\makeatletter
\patchcmd{\LT@caption}
  {\bgroup}
  {\bgroup\global\LTpatch@captiontrue}
  {}{}
\patchcmd{\longtable}
  {\par}
  {\par\global\LTpatch@captionfalse}
  {}{}
\apptocmd{\endlongtable}
  {\ifLTpatch@caption\else\addtocounter{table}{-1}\fi}
  {}{}
\newif\ifLTpatch@caption
\makeatother

\begin{document}

\maketitle


\setcounter{secnumdepth}{-\maxdimen} % remove section numbering

\setlength\LTleft{0pt}


Infants learn about the world from the people around them. An infant's
social network allows them to explore the world and learn how to connect
and communicate with others. It has been previously suggested that
children with a larger social network size are better at taking others'
perspectives (Burke et al., 2023). In the current study, we aim to
expand these findings from behavioral domain to neural domain and
examine the relation between children's social network size and their
neural responses to strangers during infancy.

\subsection{Literature Review}\label{literature-review}

\subsection{\texorpdfstring{\emph{Social Network
Size}}{Social Network Size}}\label{social-network-size}

Previous studies found that social networks play an important role in
social cognition throughout infancy and childhood (Burke et al., 2022;
Burke et al., 2023). Burke and colleagues (2023) examined the relation
between a three year-old's social network size and their
perspective-taking skills. The researchers were observing the children's
basic understanding that what they see may be different from what
someone else sees (Burke et al., 2023). The results revealed that
children with larger social networks displayed significantly better
explicit perspective-taking skills. This finding indicates that a larger
social network might improve a child's ability to consider things from
another person's point of view, or on the other hand,better perspective
taking skills can help children build a larger social network (Burke et
al., 2023).

Some indirect findings also support the role of social networks on the
development of social cognition. It has been previously found that
children aged 5-11 with at least one sibling are more likely to develop
positive peer relationships because having a sibling allows children to
have more opportunities to develop social and interpersonal skills
during early childhood (Downey \& Condron, 2004). These findings support
the close relationship between children's social network size and social
cognition. However, the connection between social network size and
social emotional development has been less examined. In addition, most
studies focused on older children, the role of social network size
during the first year of life remains unclear.

Based on these gaps in previous studies, the current study will examine
the relationship between social network size and response to strangers
of infants from 8 - 12 months of age. More specifically, we will focus
on infants' neural responses, frontal alpha asymmetry (FAA), while
watching socially interactive videos of strangers.

\subsection{\texorpdfstring{\emph{Frontal Alpha
Asymmetry}}{Frontal Alpha Asymmetry}}\label{frontal-alpha-asymmetry}

Frontal alpha asymmetry (FAA) is an important neural indicator of
infants' approach/withdrawal tendency (Davidson \& Fox, 1982; Fox \&
Davidson, 1986; Fox, 1991; Garstein et al., 2020; Harrewijn et al.,
2019; Haworth et al., 2016). FAA is the measurement of the imbalance in
cortical activation between the right and left frontal hemispheres,
measured by psycho-physiological markers using electroencephalogram
(EEG) (Davidson \& Fox, 1982; Fox \& Davidson, 1986; Fox, 1991).
Rightward/negative FAA signifies increased right hemispheric activation
as compared to left and leftward/positive FAA signifies increased left
hemispheric activation (Vincent et al., 2021). FAA is thought to be a
moderator reflecting fearful inhibition and impulsivity-anger (Liu et
al., 2021), with left FAA being associated with higher levels of
activity/approach while right FAA is associated with higher levels of
fear/withdrawal (Davidson \& Fox, 1982; Fox \& Davidson, 1986; Fox,
1991; Garstein et al., 2020; Harrewijn et al., 2019; Haworth et al.,
2016). Based on these findings, we will use FAA as a neural indicator of
infants' approach/withdrawal tendency towards strangers.

It is advantageous to utilize FAA to study the relationship between
social network size and infants' responses to strangers because neural
indicators can be more sensitive than behavioral indicators (Davidson \&
Fox, 1982; Fox \& Davidson, 1986; Fox, 1991; Harrewijn et al., 2019;
Haworth et al., 2016). The behavioral changes of 8- to 12-month-olds
might be too subtle to code, whereas neural indicators can be more
sensitive to capture individual differences, given that FAA has been
widely used to study individual differences (e.g.~motivation, affect,
depression severity, Gollan et al., 2014). Furthermore, examining the
relationship between social network size and FAA can enhance our
understanding of how social networks may shape infant's brains.

Previous studies suggest that larger social network sizes are associated
with better perspective-taking skills (Burke et al., 2023) and social
and interpersonal skills (Downey \& Condron, 2004), which indicates a
positive relationship between social network sizes and social
development. In the current study, we also expect social network sizes
to be related to more positive responses to strangers. Since positive
FAA is associated with approach tendencies in infants (Davidson \& Fox,
1982; Fox \& Davidson, 1986; Fox, 1991), we hypothesize that an infant
with a larger social network will be more likely to display more
positive FAA to strangers.

\subsection{\texorpdfstring{\emph{The Moderating Effect of
Temperament}}{The Moderating Effect of Temperament}}\label{the-moderating-effect-of-temperament}

Temperament was found to play a large role in an infant's responses to
strangers (Rubin et al., 2009). Children who are shy or display reticent
behavior will show more avoidance to strangers (Rubin et al., 2009).
Furthermore, children with different temperament will react to the same
environment differently, or in other words, temperament can moderate the
effect of the environment on child development. Ertekin and colleagues
(2021) found that infants with a more reactive temperament, including
higher levels of fear, irritability, and activity may be more sensitive
to the surrounding environment as they are more likely to be
overwhelmed. More specifically, infants with a higher rate of recovery
from distress were less impacted by adverse environments than infants
with a low rate of recovery from distress (Ertekin et al., 2021).
Another study found that temperament can moderate children's social
wariness towards strangers of the same race and different races than
their own (i.e.~`ingroup' versus `outgroup', Hwang et al., 2023). More
shy children displayed greater social wariness towards strangers of a
different race than of the same race to their own, while less shy
children did not (Hwang et al., 2023). These findings suggest that
individual differences in temperament can have a moderating effect
between social environment and children's responses to strangers.

Based on these findings, the current study will examine the moderating
role of temperament between social network size and infants' FAA to
strangers. Specifically, we will focus on the fear dimension, which
refers to the level of fearfulness or anxiousness in response to novel
or unfamiliar stimuli (Gartstein \& Rothbart, 2003), given that this
dimension is viewed as an important characteristic during the first year
of life.

\subsection{Current Study}\label{current-study}

The current study aims to examine two questions. First, this study will
examine the relationship between social network sizes and neural
responses to strangers during infancy. We used an infant's FAA as an
indicator of the infant's neural response to strangers. Given the
positive relation between social network sizes and children' social
cognition (Burke et al., 2023; Downey \& Condron, 2004), we predicted
that larger social network size will correlate with more positive FAA
(i.e.~increased left frontal activation) toward strangers.

The second question we were interested in is how temperament, especially
fear, can moderate the relationship between social network size and
responses to strangers during infancy. There are three possibilities.
First, a high level of fear may weaken the relationship between social
network size and an infant's neural response to strangers. Second, a
high level of fear may magnify the relationship between social network
size and an infant's neural response to strangers. The third possibility
is that there is no influence of fear on the relationship between social
network size and an infant's neural responses to strangers.

In summary, this study aimed to examine the connection between infants'
social network sizes and their neural response to strangers. It
contributes to the understanding of how early social environments can
shape infants' brains and how individual temperament plays a moderating
role. These factors are essential in understanding an infant's stranger
anxiety/fear and are important for caregivers to take into consideration
when scaffolding infants build their social networks.

\subsection{Methods}\label{methods}

\subsection{\texorpdfstring{\emph{Participants}}{Participants}}\label{participants}

All participants in this study were recruited by the Infant Learning and
Development Laboratory at the University of Chicago. The parents of
approximately 80 infants from 8 months to 12 months old were asked to
participate in the study. Inclusion criteria is as follows: (1) 8 to 12
months of age, (2) English language spoken at least 80\% of the time at
home, (3) full-term; 37 weeks and beyond gestational age, and (4) no
developmental delays.

\subsection{\texorpdfstring{\emph{Procedure}}{Procedure}}\label{procedure}

Our study uses an Electroencephalogram (EEG) machine and related
materials (i.e.~infant EEG cap, software, and technical equipment) to
measure the infant's neural processing of strangers, specifically alpha
wave frequency. During the EEG task, infants watched videos of two
strangers grasping a toy alternatively. One stranger speaks English, and
the other speaks French. Our analyses focus on infants' neural responses
while watching the native speaker. To calculate FAA, we subtracted the
natural log-transformed relative alpha power of the 6-9 Hz frequency
band in the left hemisphere from the natural log-transformed relative
alpha power in the right hemisphere (Anaya et al., 2021; Vincent et al.,
2021; Fox et al., 2001). Therefore, in our calculations of FAA, stronger
left frontal activation is indicated by a positive FAA score (Harrewijn
et al., 2019; Vincent et al., 2021).

In order to measure an infant's temperament and social network size,
parents were instructed to fill out the Revised Infant Behavior
Questionnaire (IBQ-R), a parent report of a reaction on a seven point
scale during the past one or two weeks, and Child Social Network
Questionnaire (CSNQ), a demographic form for each person the child sees
on a regular basis, regarding their child (Gartstein \& Rothbart, 2003;
Burke et al., 2022).

\subsection{\texorpdfstring{\emph{Analyses}}{Analyses}}\label{analyses}

This study used linear regression models to examine our hypotheses.
First, to examine the relationship between social network sizes and
infants' neural responses to strangers, we conducted a linear regression
model, with FAA while viewing the English speaking strangers as the
outcome, and social network size as the predictor. Second, to examine
the moderating effect of fear between social network sizes and infants'
responses to strangers, we conducted a moderating model with fear scores
as the moderator between social network size and infants' FAA.

\subsection{Results}\label{results}

The results presented below will accomplish the following aims. First,
we present the descriptive information about the variables. Next we aim
to answer the question: what is the relationship between social network
size and frontal alpha asymmetry in infants? And finally, we will look
at the potential moderating effect of temperament, specifically fear,
between the realtionship of social network size and FAA.

\begin{table}

{\caption{{Social Network Size and FAA Data (First 6
Rows)}{\label{tbl-snq-faa}}}
\vspace{-20pt}}

\global\setlength{\Oldarrayrulewidth}{\arrayrulewidth}

\global\setlength{\Oldtabcolsep}{\tabcolsep}

\setlength{\tabcolsep}{2pt}

\renewcommand*{\arraystretch}{1.5}



\providecommand{\ascline}[3]{\noalign{\global\arrayrulewidth #1}\arrayrulecolor[HTML]{#2}\cline{#3}}

\begin{longtable*}[c]{|p{0.75in}|p{0.75in}|p{0.75in}|p{0.75in}|p{0.75in}}



\ascline{0.75pt}{000000}{1-5}

\multicolumn{1}{>{\centering}m{\dimexpr 0.75in+0\tabcolsep}}{\textcolor[HTML]{000000}{\fontsize{11}{22}\selectfont{\global\setmainfont{Times New Roman}{ID}}}} & \multicolumn{1}{>{\centering}m{\dimexpr 0.75in+0\tabcolsep}}{\textcolor[HTML]{000000}{\fontsize{11}{22}\selectfont{\global\setmainfont{Times New Roman}{SNQ}}}} & \multicolumn{1}{>{\centering}m{\dimexpr 0.75in+0\tabcolsep}}{\textcolor[HTML]{000000}{\fontsize{11}{22}\selectfont{\global\setmainfont{Times New Roman}{Condition}}}} & \multicolumn{1}{>{\centering}m{\dimexpr 0.75in+0\tabcolsep}}{\textcolor[HTML]{000000}{\fontsize{11}{22}\selectfont{\global\setmainfont{Times New Roman}{Time}}}} & \multicolumn{1}{>{\centering}m{\dimexpr 0.75in+0\tabcolsep}}{\textcolor[HTML]{000000}{\fontsize{11}{22}\selectfont{\global\setmainfont{Times New Roman}{FAA}}}} \\

\ascline{0.75pt}{000000}{1-5}\endfirsthead 

\ascline{0.75pt}{000000}{1-5}

\multicolumn{1}{>{\centering}m{\dimexpr 0.75in+0\tabcolsep}}{\textcolor[HTML]{000000}{\fontsize{11}{22}\selectfont{\global\setmainfont{Times New Roman}{ID}}}} & \multicolumn{1}{>{\centering}m{\dimexpr 0.75in+0\tabcolsep}}{\textcolor[HTML]{000000}{\fontsize{11}{22}\selectfont{\global\setmainfont{Times New Roman}{SNQ}}}} & \multicolumn{1}{>{\centering}m{\dimexpr 0.75in+0\tabcolsep}}{\textcolor[HTML]{000000}{\fontsize{11}{22}\selectfont{\global\setmainfont{Times New Roman}{Condition}}}} & \multicolumn{1}{>{\centering}m{\dimexpr 0.75in+0\tabcolsep}}{\textcolor[HTML]{000000}{\fontsize{11}{22}\selectfont{\global\setmainfont{Times New Roman}{Time}}}} & \multicolumn{1}{>{\centering}m{\dimexpr 0.75in+0\tabcolsep}}{\textcolor[HTML]{000000}{\fontsize{11}{22}\selectfont{\global\setmainfont{Times New Roman}{FAA}}}} \\

\ascline{0.75pt}{000000}{1-5}\endhead



\multicolumn{1}{>{\centering}m{\dimexpr 0.75in+0\tabcolsep}}{\textcolor[HTML]{000000}{\fontsize{11}{22}\selectfont{\global\setmainfont{Times New Roman}{PB\_10}}}} & \multicolumn{1}{>{\centering}m{\dimexpr 0.75in+0\tabcolsep}}{\textcolor[HTML]{000000}{\fontsize{11}{22}\selectfont{\global\setmainfont{Times New Roman}{11}}}} & \multicolumn{1}{>{\centering}m{\dimexpr 0.75in+0\tabcolsep}}{\textcolor[HTML]{000000}{\fontsize{11}{22}\selectfont{\global\setmainfont{Times New Roman}{EnTT}}}} & \multicolumn{1}{>{\centering}m{\dimexpr 0.75in+0\tabcolsep}}{\textcolor[HTML]{000000}{\fontsize{11}{22}\selectfont{\global\setmainfont{Times New Roman}{-1000\ to\ 0}}}} & \multicolumn{1}{>{\centering}m{\dimexpr 0.75in+0\tabcolsep}}{\textcolor[HTML]{000000}{\fontsize{11}{22}\selectfont{\global\setmainfont{Times New Roman}{0.005574083}}}} \\





\multicolumn{1}{>{\centering}m{\dimexpr 0.75in+0\tabcolsep}}{\textcolor[HTML]{000000}{\fontsize{11}{22}\selectfont{\global\setmainfont{Times New Roman}{PB\_10}}}} & \multicolumn{1}{>{\centering}m{\dimexpr 0.75in+0\tabcolsep}}{\textcolor[HTML]{000000}{\fontsize{11}{22}\selectfont{\global\setmainfont{Times New Roman}{11}}}} & \multicolumn{1}{>{\centering}m{\dimexpr 0.75in+0\tabcolsep}}{\textcolor[HTML]{000000}{\fontsize{11}{22}\selectfont{\global\setmainfont{Times New Roman}{EnTT}}}} & \multicolumn{1}{>{\centering}m{\dimexpr 0.75in+0\tabcolsep}}{\textcolor[HTML]{000000}{\fontsize{11}{22}\selectfont{\global\setmainfont{Times New Roman}{0\ to\ 1000}}}} & \multicolumn{1}{>{\centering}m{\dimexpr 0.75in+0\tabcolsep}}{\textcolor[HTML]{000000}{\fontsize{11}{22}\selectfont{\global\setmainfont{Times New Roman}{0.020131267}}}} \\





\multicolumn{1}{>{\centering}m{\dimexpr 0.75in+0\tabcolsep}}{\textcolor[HTML]{000000}{\fontsize{11}{22}\selectfont{\global\setmainfont{Times New Roman}{PB\_11}}}} & \multicolumn{1}{>{\centering}m{\dimexpr 0.75in+0\tabcolsep}}{\textcolor[HTML]{000000}{\fontsize{11}{22}\selectfont{\global\setmainfont{Times New Roman}{17}}}} & \multicolumn{1}{>{\centering}m{\dimexpr 0.75in+0\tabcolsep}}{\textcolor[HTML]{000000}{\fontsize{11}{22}\selectfont{\global\setmainfont{Times New Roman}{EnTT}}}} & \multicolumn{1}{>{\centering}m{\dimexpr 0.75in+0\tabcolsep}}{\textcolor[HTML]{000000}{\fontsize{11}{22}\selectfont{\global\setmainfont{Times New Roman}{-1000\ to\ 0}}}} & \multicolumn{1}{>{\centering}m{\dimexpr 0.75in+0\tabcolsep}}{\textcolor[HTML]{000000}{\fontsize{11}{22}\selectfont{\global\setmainfont{Times New Roman}{0.037043985}}}} \\





\multicolumn{1}{>{\centering}m{\dimexpr 0.75in+0\tabcolsep}}{\textcolor[HTML]{000000}{\fontsize{11}{22}\selectfont{\global\setmainfont{Times New Roman}{PB\_11}}}} & \multicolumn{1}{>{\centering}m{\dimexpr 0.75in+0\tabcolsep}}{\textcolor[HTML]{000000}{\fontsize{11}{22}\selectfont{\global\setmainfont{Times New Roman}{17}}}} & \multicolumn{1}{>{\centering}m{\dimexpr 0.75in+0\tabcolsep}}{\textcolor[HTML]{000000}{\fontsize{11}{22}\selectfont{\global\setmainfont{Times New Roman}{EnTT}}}} & \multicolumn{1}{>{\centering}m{\dimexpr 0.75in+0\tabcolsep}}{\textcolor[HTML]{000000}{\fontsize{11}{22}\selectfont{\global\setmainfont{Times New Roman}{0\ to\ 1000}}}} & \multicolumn{1}{>{\centering}m{\dimexpr 0.75in+0\tabcolsep}}{\textcolor[HTML]{000000}{\fontsize{11}{22}\selectfont{\global\setmainfont{Times New Roman}{0.019238154}}}} \\





\multicolumn{1}{>{\centering}m{\dimexpr 0.75in+0\tabcolsep}}{\textcolor[HTML]{000000}{\fontsize{11}{22}\selectfont{\global\setmainfont{Times New Roman}{PB\_13}}}} & \multicolumn{1}{>{\centering}m{\dimexpr 0.75in+0\tabcolsep}}{\textcolor[HTML]{000000}{\fontsize{11}{22}\selectfont{\global\setmainfont{Times New Roman}{3}}}} & \multicolumn{1}{>{\centering}m{\dimexpr 0.75in+0\tabcolsep}}{\textcolor[HTML]{000000}{\fontsize{11}{22}\selectfont{\global\setmainfont{Times New Roman}{EnTT}}}} & \multicolumn{1}{>{\centering}m{\dimexpr 0.75in+0\tabcolsep}}{\textcolor[HTML]{000000}{\fontsize{11}{22}\selectfont{\global\setmainfont{Times New Roman}{0\ to\ 1000}}}} & \multicolumn{1}{>{\centering}m{\dimexpr 0.75in+0\tabcolsep}}{\textcolor[HTML]{000000}{\fontsize{11}{22}\selectfont{\global\setmainfont{Times New Roman}{-0.018925968}}}} \\





\multicolumn{1}{>{\centering}m{\dimexpr 0.75in+0\tabcolsep}}{\textcolor[HTML]{000000}{\fontsize{11}{22}\selectfont{\global\setmainfont{Times New Roman}{PB\_13}}}} & \multicolumn{1}{>{\centering}m{\dimexpr 0.75in+0\tabcolsep}}{\textcolor[HTML]{000000}{\fontsize{11}{22}\selectfont{\global\setmainfont{Times New Roman}{3}}}} & \multicolumn{1}{>{\centering}m{\dimexpr 0.75in+0\tabcolsep}}{\textcolor[HTML]{000000}{\fontsize{11}{22}\selectfont{\global\setmainfont{Times New Roman}{EnTT}}}} & \multicolumn{1}{>{\centering}m{\dimexpr 0.75in+0\tabcolsep}}{\textcolor[HTML]{000000}{\fontsize{11}{22}\selectfont{\global\setmainfont{Times New Roman}{-1000\ to\ 0}}}} & \multicolumn{1}{>{\centering}m{\dimexpr 0.75in+0\tabcolsep}}{\textcolor[HTML]{000000}{\fontsize{11}{22}\selectfont{\global\setmainfont{Times New Roman}{-0.024349337}}}} \\

\ascline{0.75pt}{000000}{1-5}



\end{longtable*}



\arrayrulecolor[HTML]{000000}

\global\setlength{\arrayrulewidth}{\Oldarrayrulewidth}

\global\setlength{\tabcolsep}{\Oldtabcolsep}

\renewcommand*{\arraystretch}{1}

{\vspace{-20pt}
\noindent \emph{Note.} Table 1 displays social network size (SNQ) and FAA data for the first three participants. Each participant has FAA measured for two conditions, -1000 to 0 and 0 to 1000.}

\end{table}

The statistics below detail a 1-way ANOVA between social network size
and FAA.

\begin{verbatim}
             Df Sum Sq Mean Sq F value  Pr(>F)   
FAA           1    280  280.50   10.65 0.00135 **
Residuals   158   4163   26.35                   
---
Signif. codes:  0 '***' 0.001 '**' 0.01 '*' 0.05 '.' 0.1 ' ' 1
\end{verbatim}

Figure 1 below demonstrates the relationship between social network size
and FAA for each participant during the two time periods. As we can see,
FAA has a tendency to increase as social network size increases. This
supports our hypothesis that a larger social network size will lead to
more positive FAA.

\begin{verbatim}
`geom_smooth()` using formula = 'y ~ x'
\end{verbatim}

\begin{figure}[!htbp]

{\caption{{Social Network Size and FAA by Time
Blocks}{\label{fig-scatter-2}}}}

\pandocbounded{\includegraphics[keepaspectratio]{SNQ-FAA-Data_files/figure-pdf/fig-scatter-2-1.pdf}}

{\noindent \emph{Note.} Figure 1 displays levels of FAA and infants'
social network sizes. Each participant has two occurences, one in each
of the time blocks.}

\end{figure}

\subsection{Discussion}\label{discussion}

The aim of our study was to answer two questions: what is the
realtionship between social network size and FAA in infants towards
strangers and what is the moderating effect of fear in this
relationship?

We have found evidence to support our first hypothesis, that a larger
social network size is associated with more positive FAA while viewing
the videos of strangers.

Therefore, we are be further interested in the moderating effect of fear
in this relationship. The findings of this study will contribute to the
understanding of the neurological reactions, and affecting factors,
toward strangers during infancy.






\end{document}
